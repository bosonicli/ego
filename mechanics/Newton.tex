\documentclass[aps,preprint,amsmath,amssymb,nofootinbib,superscriptaddress,floatfix]{revtex4-1}
\usepackage{graphicx}
\usepackage{tabularx}
\usepackage{subfigure}
\usepackage{dcolumn}
\usepackage{bm}
\usepackage{euscript}
%\usepackage{ulem}
\usepackage{color}
\usepackage{epsfig,psfrag,subfigure}
\usepackage{amsfonts}
\usepackage{exscale}
\usepackage{amsbsy}
\usepackage{placeins}
\usepackage{CJK}
%\usepackage{ulem}
%\usepackage{babel}
%\usepackage{times.sty}

\def\avg#1{\left\langle#1\right\rangle}
\def\bra#1{\left\langle#1\right|}
\def\ket#1{\left|#1\right\rangle}
\def\braket#1#2{\left\langle #1\right|\left.#2\right\rangle}
\def\combin#1#2{\kc{^{#1}_{#2}}}
\def\abs#1{\left|#1\right|}
\def\kc#1{\left(#1\right)}
\def\kd#1{\left[#1\right]}
\def\ke#1{\left\{#1\right\}}
\def\Re{{\rm Re}}
\def\Im{{\rm Im}}
\def\sgn{{\rm sgn}}
\def\mode{\text{ }{\rm mod}\text{ }}
\def\trace#1{{\rm Tr}\left[#1\right]}
\def\Det#1{{\rm Det}\left(#1\right)}
\def\be{\begin{equation}}       \def\ee{\end{equation}}
\def\bea{\begin{eqnarray}}      \def\eea{\end{eqnarray}}
\def\ba{\begin{array}}
\def\ea{\end{array}}
\def\bnum{\begin{enumerate} }
\def\enum{\end{enumerate}}
\def\lf{\left}
\def\rt{ \right}
\def\nn{\nonumber}
\def\pa{\partial}
\def\=>{\Rightarrow}
\def\>{\rightarrow}
\def\A{\uparrow}
\def\V{\downarrow}
\def\PRB{Phys. Rev. B}
\def\PRL{Phys. Rev. Lett.}
\def\eye2{Fathbb{I}}
\def\vk{{\bf k}}
\def\Tau{Fathcal{T}}
\def\Ep{Fathcal{E}}
\def\Kp{Fathcal{K}}
\def\intz{Fathbb{Z}}
\def\vect#1{\kc{\ba{c}#1\ea}}
\def\elist#1{\left\{\ba{cc} #1\ea\right.}
\def\Cl{{\rm Cliff}}
\def\Pf#1{{\rm Pf}\kc{#1}}
\def\m{\textrm{matter}}
\def\g{\textrm{gauge}}

\def\Eq#1{Eq.~(\ref{#1})}
\def\Fig#1{Fig.~\ref{#1}}

\def\te{\mathrm{e}}
\def\eff{\mathrm{eff}}
\def\tr{\mathrm{tr}}
\def\Tr{\mathrm{Tr}}
\def\bS{{\bf S}}
\def\tS{{\tilde S}}
\def\tP{{\tilde P}}

\renewcommand{\v}[1]{{\bf #1}}
\renewcommand{\c}[1]{{\cal #1}}
\renewcommand{\b}[1]{{\bar{ #1}}}
\newcommand{\w}{{\omega}}
\newcommand{\s}{{\sigma}}
\newcommand{\bz}{\bar{z}}
\newcommand{\gr}{{\nabla}}
%\newcommand{\Ref}[1]{Ref.~\cite{#1}}
\newcommand{\no}{\nonumber}
\newcommand{\<}{\langle}
\renewcommand{\>}{\rangle}
\renewcommand{\Im}{{\rm Im}}
\renewcommand{\Re}{{\rm Re}}
\newcommand{\p}{\partial}
\renewcommand{\t}[1]{{\tilde #1}}
\newcommand{\ua}{\uparrow}
\newcommand{\da}{\downarrow}
\newcommand{\ra}{\rightarrow}
\newcommand{\lra}{\leftrightarrow}
\newcommand{\del}{\delta}
\newcommand{\Del}{\Delta}
\newcommand{\e}{\epsilon}
\newcommand{\ga}{\gamma}
\newcommand{\Ga}{\Gamma}
\newcommand{\ka}{\kappa}
\newcommand{\la}{\lambda}
\newcommand{\La}{\Lambda}
\newcommand{\Om}{\Omega}
\renewcommand{\th}{\theta}
\newcommand{\Th}{\Theta}
\newcommand{\Si}{\Sigma}

\newcommand {\C}{\textcolor {red}}
\newcommand {\B}{\textcolor {blue}}

\newcommand{\rem}[1]{\sout{\textcolor{magenta}{#1}}}
\newcommand{\rep}[2]{\sout{\textcolor{magenta}{#1}}\ \textcolor{blue}{#2}}
\newcommand{\ins}[1]{\textcolor{blue}{#1}}

\usepackage{graphicx}% Include figure files
\usepackage{dcolumn}% Align table columns on decimal point
\usepackage{bm}% bold math
%\usepackage{hyperref}% add hypertext capabilities
%\usepackage[mathlines]{lineno}% Enable numbering of text and display math
%\linenumbers\relax % Commence numbering lines

%\usepackage[showframe,%Uncomment any one of the following lines to test
%%scale=0.7, marginratio={1:1, 2:3}, ignoreall,% default settings
%%text={7in,10in},centering,
%%margin=1.5in,
%%total={6.5in,8.75in}, top=1.2in, left=0.9in, includefoot,
%%height=10in,a5paper,hmargin={3cm,0.8in},
%]{geometry}

\usepackage{braket}
\usepackage{mathrsfs}
\usepackage{hyperref}

\newcommand{\norm}[1]{\vert #1 \vert}
\renewcommand{\vec}[1]{\boldsymbol{#1}}


%\bibliographystyle{apsrev4-1}
\begin{document}

%\preprint{APS/123-QED}
\graphicspath{{../../diagram_repository/}}
\title{Extended-Kitaev $J_2-J_3$ model}
\author{Bo-Hai Li}

%\maketitle

\appendix
\renewcommand\thefigure{\Alph{section}\arabic{figure}}
\renewcommand\theequation{\Alph{section}\arabic{equation}}
\renewcommand\thetable{\Alph{section}\arabic{table}}

\tableofcontents

% !TeX root = Newton.tex

\section{\label{sec:Kepler}Kepler Problem}

\subsection{\label{sec:Kepler_Category}Category}

	\begin{itemize}
		\item
		Effective potential
		\item
		Stability
		\item
		$O(4)$ Symmetry
		\item
		Bertland Theorem
	\end{itemize}

% !TeX root = Newton.tex

\section{\label{sec:Mec}Mechanics, symmetries, currents}

\subsection{\label{sec:SO4}Hydrogen and SO(4) symmetry}

	Laplace-Runge-Lenz vector does not necessarily commute with angular momentum $L$ in the $\mathcal{L}$ notion. It is conserved given that $\mathcal{H}$ is time-independent. In other words, it is invariant under a different (more strict) variational condition.

\subsection{\label{sec:Action}Action and time evolution}

	Time evolution operators classically (failed)

\subsection{\label{sec:Bertland}Orbit stability}

	Bertland Theorem

% !TeX root = Newton.tex

\section{\label{sec:NonInertial}Non-inertial system}

\subsection{\label{sec:Tide}Tidal Force}

	\bea
		(1+x)^{-\frac{1}{2}} &=& 1 + (-\frac{1}{2} ) x + (\frac{3}{8}) x^2 + o(x^3)	\label{eqn:Taylor_M12}	\\
		(1+x)^{-\frac{3}{2}} &=& 1 + (-\frac{3}{2} ) x + (\frac{15}{8}) x^2 + o(x^3)	\label{eqn:Taylor_M32}
	\eea

	\bea
		\vec{a}_{g} &=& - \frac{k}{\norm{\vec{r}}^3} \vec{r}	\nn	\\
		&=& - k (\vec{r}^2)^{-\frac{3}{2}} \vec{r}	\nn	\\
		&=& - \frac{k}{r_0^3} (1 + \frac{(\vec{r_0}+\vec{\Delta r})^2-\vec{r_0}^2}{\vec{r_0}^2})^{-\frac{3}{2}} (\vec{r_0} + \vec{\Delta r})	\nn	\\
		&=& - \frac{k}{r_0^3} (1 + \frac{2 \vec{r_0} \cdot \vec{\Delta r} + (\vec{\Delta r})^2}{\vec{r_0}^2})^{-\frac{3}{2}} (\vec{r_0} + \vec{\Delta r})	\nn	\\
		&=& - \frac{k}{r_0^3} (1 - \frac{3 \vec{r_0} \cdot \vec{\Delta r}}{\vec{r_0}^2} + \frac{15 (\vec{r_0} \cdot \vec{\Delta r})^2}{2 \vec{r_0}^4} - \frac{3 (\vec{\Delta r})^2}{2 (\vec{r_0})^2}) (\vec{r_0} + \vec{\Delta r})	\label{eqn:a_g}
	\eea

	\bea
		\vec{a}_{e} &=& - \frac{k}{r_0^3} \vec{r_0}	\nn	\\
		\vec{a}_{Tide} &=& \vec{a}_{g} - \vec{a}_{e}	\nn	\\
		&=& o((\vec{\Delta r})^3) - \frac{k}{r_0^3} (\vec{\Delta r} - \frac{3 \vec{r_0} \cdot \vec{\Delta r}}{\vec{r_0}^2} \vec{r_0})	\nn	\\
		&-& \frac{k}{r_0^3} ( \frac{15 (\vec{r_0} \cdot \vec{\Delta r})^2}{2 \vec{r_0}^4} - \frac{3 (\vec{\Delta r})^2}{2 (\vec{r_0})^2}) \vec{r_0} - \frac{k}{r_0^3} (- \frac{3 \vec{r_0} \cdot \vec{\Delta r}}{\vec{r_0}^2}) \vec{\Delta r}	\nn	\\
		&=& \vec{a}_{Tide}^{1} + \vec{a}_{Tide}^{2} + o((\vec{\Delta r})^3)	\label{eqn:a_Tide}	\\
		\vec{a}_{Tide}^{1} &=& - \frac{k}{r_0^3} (\vec{\Delta r}_{\perp} - 2 \vec{\Delta r}_{\parallel})	\label{eqn:a_Tide_1}
	\eea

	\bea
		V_{g} &=& - \frac{k}{\norm{\vec{r}}}	\nn	\\
		&=& - k (\vec{r}^2)^{-\frac{1}{2}}	\nn	\\
		&=& - \frac{k}{r_0} (1 + \frac{(\vec{r_0}+\vec{\Delta r})^2-\vec{r_0}^2}{\vec{r_0}^2})^{-\frac{1}{2}}	\nn	\\
		&=& - \frac{k}{r_0} (1 + \frac{2 \vec{r_0} \cdot \vec{\Delta r} + (\vec{\Delta r})^2}{\vec{r_0}^2})^{-\frac{1}{2}}	\nn	\\
		&=& - \frac{k}{r_0} (1 - \frac{ \vec{r_0} \cdot \vec{\Delta r}}{\vec{r_0}^2} + \frac{3 (\vec{r_0} \cdot \vec{\Delta r})^2}{2 \vec{r_0}^4} - \frac{ (\vec{\Delta r})^2}{2 (\vec{r_0})^2}) + o((\vec{\Delta r})^3)	\nn	\\
		&=& V_{0} + V_{1} + V_{2} + o((\vec{\Delta r})^3)	\label{eqn:V_g}
	\eea

\subsection{\label{sec:Coriolis}Coriolis Force}

	In system with rotation $(\vec{\omega}, \vec{\beta})$ around point $O$, Non-inertial forces are

	\bea
		\vec{a}_{c} &=& -2 (\vec{\omega} \times \vec{\delta v})	\nn	\\
		\vec{a}_{\beta} &=& - \beta \times \vec{\delta r}	\label{eqn:Coriolis}
	\eea

% !TeX root = Newton.tex

\section{\label{sec:Thermal}Thermaldynamics and Hydrodynamics}

\subsection{\label{sec:Roche}Fluid Roche limit}

	To be continued.

\subsection{\label{sec:Waterball}Waterball without gravity}

	Ossilation?

% !TeX root = Newton.tex

\section{\label{sec:Scenario}Scenarios}

\subsection{\label{sec:Wander}Wander around space station}

	Space station $O$ is orbiting $\vec{r}:(r,\theta)$ around the earth and an astronaut is wandering around $O$ with displacement $\vec{\Delta r}$ in non-rotating system and $\vec{\delta r}$ in rorating system.

	Dynamic in the rotating system is described in Eqn.~\ref{eqn:a_Wander}

	\begin{eqnarray}
		{\vec{a}}_{\delta} &=& \vec{a}_{Tide} + \vec{a}_{c} + \vec{a}_{\beta}	\nn	\\
		\vec{a}_{Tide} &=& - \frac{k}{r_0^3} (- 2 \delta r_{r} \hat{r} + \delta r_{\theta} \hat{\theta} )	\nn	\\
		\vec{a}_{c} &=& 2 \dot{\theta} (\dot{\delta r_{\theta}} \hat{r} - \dot{\delta r_{r}} \hat{\theta} )	\nn	\\
		\vec{a}_{\beta} &=& \ddot{\theta} (\delta r_{\theta} \hat{r} - \delta r_{r} \hat{\theta} ) 	\label{eqn:a_Wander}
	\end{eqnarray}

	Assume $O$ is orbiting on a circle $(\dot{\theta}, \ddot{\theta}) = (\omega,0)$, we have $\omega = \frac{k}{r_0^3}$, then Eqn.~\ref{eqn:a_Wander} is simplified as Eqn.~\ref{eqn:a_Wander_Circle}

	\begin{eqnarray}
		\frac{d^2}{dt^2} \vec{\delta r} &=& \vec{a}_{Tide} + \vec{a}_{c}	\nn	\\
		&=& -\omega^2 (- 2 \delta r_{r} \hat{r} + \delta r_{\theta} \hat{\theta} ) + 2 \omega (\dot{\delta r_{\theta}} \hat{r} - \dot{\delta r_{r}} \hat{\theta} )	\label{eqn:a_Wander_Circle}
	\end{eqnarray}

	Qualitative analysis of the dynamics: Assume the astronaut orbits around the space station with $\dot{\delta \theta} < 0$ and a period same as the space station orbit $T_0$

	\begin{itemize}
		\item
		Averagely, orbit of $\vec{\delta r}$ operates with $\vec{-\omega}$;
		\item
		At vertex along the $\hat{r}$ direction, $\vec{a}_{Tide}$ points outwards, so velocity should be large to generate massive $\vec{a}_{c}$;
		\item
		At vertex along the $\hat{\theta}$ direction, $\vec{a}_{Tide}$ points inwards, enough to keep the orbit bound, so velocity should be small;
		\item
		Quantitave description of the orbit dynamic remains mystery;
	\end{itemize}

\subsection{\label{sec:DysonSphere}Effective gravity induced by rotation of facility}

	Only second-order effect is present at a perturbative level.

\subsection{\label{sec:Lagrangian}Lagrangian points}

	We consider the effective dynamics in a Non-inertial system of two-body gravity system.

	Consider a two-body system consists of two celestial bodies $M_1$ and $M_2$ with distance of $D$. The two-body effective mass and orbiting angular velocity are

	\begin{eqnarray}
		M &=& \frac{M_1 M_2}{M_1+M_2}	\nn	\\
		\frac{G M_1 M_2}{D^2} &=& \frac{M_1 M_2} \omega^2 D	\nn	\\
		\omega^2 &=& \frac{G (M_1 + M_2)}{D^3}
	\end{eqnarray}

	The two celestial bodies are orbiting around the centroid $O$. Now we consider the Lagrangian point $L_4$ locating at the vertex of an equilateral triangle connecting $M_1$ and $M_2$, thus we have $\vec{r} = \frac{M_1 \vec{r_1} + M_2 \vec{r_2}}{M_1 + M_2}$

	In the rotating celestial system, the effective potential around $L_4$ is

	\begin{eqnarray}
		V_{eff} &=& V_0(\vec{r_1}) + V_1(\vec{r_1}) + V_2(\vec{r_1}) + V_0(\vec{r_2}) + V_1(\vec{r_2}) + V_2(\vec{r_2}) + o((\vec{\Delta r})^3) + V_{\omega}	\nn	\\
		&=& V_{eff}^{0} + V_{eff}^{1} + V_{eff}^{2} + o((\vec{\Delta r})^3)	\nn	\\
		V_{\omega} &=& -\frac{1}{2} \omega^2 (\vec{r} + \vec{\Delta r})^2 \nn	\\
		V_{eff}^{1} &=& -\frac{G M_1}{D} (-\frac{\vec{r_1} \cdot \vec{\Delta r}}{\norm{\vec{r_1}}^2})  -\frac{G M_2}{D} (-\frac{\vec{r_2} \cdot \vec{\Delta r}}{\norm{\vec{r_2}}^2}) - \omega^2 \vec{r} \cdot \vec{\Delta r}	\nn	\\
		&=& \frac{G}{D^3} (M_1 \vec{r_1} + M_2 \vec{r_2}) \cdot \vec{\Delta r} - \frac{G}{D^3} (M_1 \vec{r_1} + M_2 \vec{r_2}) \cdot \vec{\Delta r}	\nn	\\
		&=& 0	\nn	\\
		V_{eff} &=& V_{eff}^{0} + V_{eff}^{2} + o((\vec{\Delta r})^3)	\label{eqn:V_L4}
	\end{eqnarray}

	In fact $V_{eff}^2$ is convex around $L_4$, thus $L_4$ is a smooth maximum in the rotating system. Celestial bodies around $L_4$ are bounded by Coriolis force under certain condition (the mass ratio boundary $\frac{M_1}{M_2}$ actually)

\subsection{\label{sec:Epicycle}Epicycle model}

	Fitting Kepler orbit of binary star system to epicycle system?

\subsection{\label{sec:Channel}Space channel in gravitational fields}

	More fictional rather than a realistic one. Few materials are found.

\bibliography{../../ref_repository/EndNote_Bohai_2021Sep}

\end{document}
%
% ****** End of file apssamp.tex ******
